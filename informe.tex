\documentclass[a4paper]{report}

% Paquetes
% CUIDADO: alterar el orden puede generar errores inesperados
% ============================================================================ %
\usepackage[spanish]{babel} % Paquete para utilizar el idioma español
\usepackage[utf8]{inputenc} % Codificación de caracteres UTF-8
\usepackage{setspace} % Para ajustar el espaciado entre líneas
\usepackage{titlesec} % Para personalizar los encabezados de capítulos y secciones
\usepackage{lipsum} % Para generar texto ficticio, puedes eliminar esta línea
\usepackage{enumitem} % Para enumerar items
\usepackage[hidelinks]{hyperref} % Para generar los enlaces a las secciones correspondientes
\usepackage{geometry} % Permite ajustar y modificar los márgenes y dimensiones de la página en LaTeX.
\usepackage{graphicx} % Proporciona comandos para incluir y manipular imágenes en LaTeX.
\usepackage{fancyhdr} % Permite personalizar el encabezado y pie de página en LaTeX.
\usepackage{microtype} % Mejora la apariencia del texto ajustando el espaciado entre caracteres y palabras.
\usepackage{chngcntr} % Permite cambiar el formato y numeración de los contadores en LaTeX.
\usepackage{tocbibind} % Controla la inclusión de elementos como la tabla de contenido, lista de figuras y lista de tablas en el índice general.
\usepackage{array} % Amplía las capacidades de las tablas en LaTeX, permitiendo personalizar el formato y comportamiento de las celdas.
\usepackage{cellspace} % Proporciona espacios adicionales en las celdas de las tablas para mejorar su apariencia.
\usepackage[acronym]{glossaries} % Facilita la creación y gestión de glosarios y listas de acrónimos en LaTeX.
\usepackage{float} % Proporciona mejoras en el posicionamiento de objetos flotantes, como figuras y tablas, en LaTeX.
\usepackage{fontspec} % Permite la selección y configuración de fuentes tipográficas en LaTeX.
\usepackage{amsmath} % Mejora la calidad tipográfica de las fórmulas matemáticas y proporciona comandos y entornos adicionales.
\usepackage{xcolor} % Amplía las capacidades de color en LaTeX, permitiendo el uso de colores personalizados.
\usepackage{listings} % Facilita la inclusión y formateo de código fuente en documentos LaTeX.
\usepackage[all]{hypcap} % Ajusta el comportamiento de los enlaces generados por las referencias a objetos flotantes como figuras y tablas
\usepackage{multicol} % Se utiliza para crear dos columnas
% ============================================================================ %

% Otras configuraciones del preámbulo

% Header y footer
\pagestyle{fancy}
\fancyhf{}
\lhead{Procesamiento Digital de Señales}
\rhead{Trabajo Integrador}
\lfoot{Boeri - Campero - Villafañe | Ingeniería en Computación}
\rfoot{\thepage}

\fancypagestyle{plain}{
  \fancyhf{}
  \lhead{Procesamiento Digital de Señales}
  \rhead{Trabajo Integrador}
  \lfoot{Boeri - Campero - Villafañe | Ingeniería en Computación}
  \rfoot{\thepage}
}

\fancyhfoffset[R]{0pt} % Ajusta el ancho del encabezado
\fancyhfoffset[L]{0pt} % Ajusta el ancho del pie de página

\newgeometry{
  top=1.25in,
  bottom=1in,
  outer=0.75in,
  inner=0.75in,
}


% Configuración de página
\setlength{\parindent}{0pt} % Deshabilitar sangría al inicio de párrafos
\setlength{\parskip}{1em} % Establecer espaciado entre párrafos

% Personalizaciones
\titleformat{\chapter}[display]
{\normalfont\huge\bfseries}{\chaptertitlename\ \thechapter}{20pt}{\Huge}
\titlespacing*{\chapter}{0pt}{0pt}{40pt} % Ajustar espaciado antes y después de los encabezados de capítulo

\renewcommand{\normalsize}{\fontsize{14}{18}\selectfont}

\newcommand{\titulo}{Sistema de Transmisión AM con Codificación DTMF}
\newcommand{\subtitulo}{Informe de Trabajo Integrador}
\newcommand{\materia}{Procesamiento Digital de Señales}
\newcommand{\departamento}{Cátedra de Procesamiento de Señales}
\newcommand{\unidadacademica}{Facultad de Ciencias Exactas y Tecnología}
\newcommand{\universidad}{Universidad Nacional de Tucumán}
\newcommand{\autores}{
  Boeri, Benjamin \\
  Campero, Leandro \\
  Villafañe, Cristian
}
\newcommand{\fecha}{\today}

% Comienza el documento
\begin{document}

% Página de título
\begin{titlepage}
  \centering
  \vspace*{1cm}

  \begin{spacing}{2}
    \textbf{\Huge \titulo}
  \end{spacing}

  \Large \subtitulo

  \vspace{1cm}

  \textbf{\Large \materia}

  \vspace{0.5cm}

  \textbf{\Large \departamento}

  \vspace{0.5cm}

  \textbf{\Large \unidadacademica}

  \vspace{0.5cm}

  \textbf{\Large \universidad}

  \vspace{2cm}

  \textbf{\Large Autores:}

  \autores

  \vfill

  \Large \fecha

\end{titlepage}

% Resumen
\begin{abstract}
  El procesamiento digital de señales se refiere a la manipulación, análisis y transformación de señales utilizando algoritmos y técnicas computacionales. En este proyecto, el procesamiento digital de señales se aplica para generar y decodificar los tonos DTMF, así como para simular la transmisión y detección de los dígitos enviados.

  Este proyecto presenta la implementación de un sistema de Modulación en Amplitud (AM) y Codificación DTMF utilizando MATLAB/SIMULINK. Se busca transmitir dígitos numéricos codificados en DTMF a través de un enlace cableado simulado. El sistema involucra la generación de tonos DTMF, el diseño de filtros digitales pasa bandas para el decodificador, la configuración de la frecuencia de portadora RF y la modelización del canal de transmisión como un filtro analógico pasa banda.

  Un filtro digital es un componente esencial en el procesamiento digital de señales que permite modificar las características de una señal. En este proyecto, se utilizan filtros digitales pasa bandas implementados mediante la técnica del filtro Butterworth. Estos filtros permiten seleccionar y aislar las frecuencias específicas asociadas a los tonos DTMF.

  La implementación de los filtros digitales se logra mediante la transformación de los coeficientes del filtro en una representación numérica que se aplica a la señal de entrada. Esto puede lograrse utilizando algoritmos y técnicas de programación, así como también herramientas como MATLAB y SIMULINK.

  El resultado de este proyecto demuestra la viabilidad y efectividad de la implementación del sistema propuesto. El procesamiento digital de señales y el uso de filtros digitales son fundamentales en diversas aplicaciones, incluyendo las comunicaciones y el procesamiento de señales de audio.

  En resumen, este proyecto combina el procesamiento digital de señales, la modulación AM, el diseño de filtros digitales y la codificación DTMF para lograr la transmisión y detección de dígitos numéricos. La implementación exitosa de este sistema contribuye al avance y comprensión de las técnicas de procesamiento y transmisión de señales en el ámbito de las comunicaciones.
\end{abstract}


% Índice
\tableofcontents

% Introduccion
% Capítulo 1
\chapter{Introducción}
\section{Problema propuesto}
La modulación en amplitud o \gls{am}, permite la transmisión de una señal de baja frecuencia superpuesta a una onda de alta frecuencia. Este sistema de modulación permite enviar mensajes en la forma de envolventes de la onda portadora, ya sea por un canal de aire o físico utilizando un enlace cableado.\\
El sistema de codificación \gls{dtfm}, utiliza una combinación de tonos de frecuencia audibles pera representar el conjunto de números del 0 al 9 disponible en el teclado telefónico, con lo cual es posible enviar una codificación numérica por la línea telefónica.\\
El modelo de trabajo está representado en la Figura \ref{fig:intro_diagrama_bloques}, correspondientes al Modulador y Demodulador \gls{am}, el canal de cable telefónico, y las etapas de codificación y decodificación \gls{dtfm}.

\begin{figure}[ht]
  \centering
  \includegraphics[width=\linewidth]{images/planteamiento/bloques.png}
  \caption{Diagrama de bloques general}
  \label{fig:intro_diagrama_bloques}
\end{figure}

\section{Objetivo}
El objetivo principal de este proyecto integrador es la implementación del sistema mostrado en la Figura \ref{fig:intro_diagrama_bloques}, utilizando MATLAB, SIMULINK, o la combinación de ambos recursos de modelado computacional, para el envío de números (0-9) codificados en \gls{dtfm} bajo modulación \gls{am}, y la detección del número enviado a la salida (uno número cada por vez).\\
A modo de referencia, el Cuadro \ref{tab:combinacion_tonos}, muestra la combinación de tonos audibles asociados al conjunto numérico, y en el enlace indicado se encuentra la información ampliada sobre la codificación \gls{dtfm}.

\begin{table}[H]
  \centering
  \begin{tabular}{|Sc|Sc|Sc|Sc|}
    \hline
    \textbf{Frecuencia Baja} & \textbf{Frecuencia Alta} & \textbf{Digito} & \textbf{Frecuencia Final} \\ \hline
    697                      & 1209                     & 1               & 1906                      \\ \hline
    697                      & 1336                     & 2               & 2033                      \\ \hline
    697                      & 1477                     & 3               & 2174                      \\ \hline
    770                      & 1209                     & 4               & 1979                      \\ \hline
    770                      & 1336                     & 5               & 2106                      \\ \hline
    770                      & 1477                     & 6               & 2247                      \\ \hline
    852                      & 1209                     & 7               & 2061                      \\ \hline
    852                      & 1336                     & 8               & 2188                      \\ \hline
    852                      & 1477                     & 9               & 2329                      \\ \hline
    941                      & 1336                     & 0               & 2277                      \\ \hline
  \end{tabular}
  \caption{Combinación de tonos audibles (medido en [Hz])}
  \label{tab:combinacion_tonos}
\end{table}

\section{Enunciado}
\begin{enumerate}[label=\alph*)]
  \item A nivel simulación se deberán sintetizar los tonos asociados a cada digito numérico seleccionando una \gls{fs} apropiada (Teorema de Nyquist-Shannon).
  \item El demodulador \gls{dtfm} deberá ser implementado mediante filtros digitales pasa bandas, con un orden y respuestas apropiadas. El modo de indicar cuál fue el digito enviado queda a criterio del grupo de trabajo.
  \item Para el modelo de trasmisión AM (enlace cableado) se deberán establecer y sintetizar la frecuencia de portadora $RF$ el índice de modulación apropiados (recordando que \gls{fs} es única en todo el sistema).
  \item El canal de transmisión se corresponde al de un filtro analógico (transformado a digital) pasa banda con un rango de 300 Hz a 3400 Hz, respuesta plana y orden apropiado. Se considera el rango útil asignado a la frecuencia telefónica, aunque el cable         telefónico de cobre tipo AWG-24, por ejemplo, supera este ancho de banda a 1Mz en distancias inferiores a 200 Mts.
\end{enumerate}

\section{Lineamientos Generales}
\begin{enumerate}[label=\alph*)]
  \item El grupo de trabajo deberá cumplir con las especificaciones del proyecto, utilizando criterios de diseños justificados para cada bloque del sistema.
  \item Se deberán indicar el paso a paso para el diseño de los filtros digitales utilizados en las diferentes etapas.
  \item El criterio de selección para el filtro analógico representativo del canal (Bessel, Butterworth, etc.), y el método de transformación analógico a discreto escogido, brindando una gráfica comparativa de la respuesta en frecuencia resultantes en ambos planos (Laplace y Z).
  \item Se pide 3 aplicaciones posibles del sistema desarrollado en aplicaciones de tele comando (por ejemplo, aplicación de sistema de riego por comando telefónico de 3 zonas), y como se imprentaría en la práctica (no el desarrollo, solo la propuesta).
  \item Problema de análisis: para el caso de que ocurran fallos en el canal de comunicación (por ejemplo, una atenuación en determinadas frecuencias), analizar la robustez del código detector para al menos 3 zonas atenuadas de frecuencias diferentes. Utilizar el código adjunto en Matlab para el diseño del canal con fallas. Justificar los resultados.
  \item Escribir el informe, Incluir conclusiones, observaciones y sugerencias sobre los resultados obtenidos
\end{enumerate}

\section{Instrucciones de simulación}
Para poder reproducir las simulaciones expuestas en este informe hay que tener en cuenta los siguientes archivos.

\subsection*{Scripts}
Dentro del directorio matlab se encuentran los scripts de las simulaciones. Existen subdirectorios que contienen algunas de las librerías propias utilizadas. A continuación se describen los archivos dentro del directorio:
\begin{enumerate}
  \item \textbf{\lstinline{sim_ideal.m}}: ejecutar este script para generar el banco de filros más el canal ideal (mayor ancho de banda).
  \item \textbf{\lstinline{sim_real.m}}: ejecutar este script para generar el banco de filros más el canal real (ancho de banda reducido, como de un canal telefónico).
  \item \textbf{\lstinline{sim_falla.m}}: ejecutar este script para generar el banco de filros más el canal con fallas.
  \item \textbf{\lstinline{analisis.m}}: script utilizado para analizar uno por uno los prototipos digitales del banco de filtros.
  \item \textbf{\lstinline{canal.m}}: script utilizado para analizar el filtro representativo del canal de transmisión. Por defecto analiza el canal con fallas pero se puede modificar.
\end{enumerate}

\subsection*{Simulink}
Dentro del directorio matlab se encuentran los archivos del Simulink, en los que se desarrolla el sistema. Existen dos archivos:
\begin{enumerate}
  \item \textbf{\lstinline{simulacion_completa.slx}}: sistema para simular toda la arquitectura completa. Contiene los selectores de tonos más un selector de canal utilizado en la simulación de canal real (telefónico).
  \item \textbf{\lstinline{simulacion_sin_am.slx}}: sistema para simular la arquitectura sin la modulación y demodulación en amplitud (más adelante se detalla la razón de esta modificación, en el Capítulo \ref{cap:desarrollo}). Contiene los mismos selectores, solo que el del canal tiene para alternar entre canal amplificado y no amplificado.
\end{enumerate}

% Planteamiento
\chapter{Planteamiento}
\section{Sistema}
Lo que se busca es simular un sistema \gls{dtfm} cuya señal se transmite a través de un canal telefónico con modulación \gls{am}. Tal simulación debe comprender cada uno de los bloques intervinientes en el sistema, como se muestra en la Figura \ref{fig:diagrama_bloques_objetivo}. Antes de diseñar y planificar la simulación hay que tener en cuenta las frecuencias intervinientes, particularmente hablando de la \gls{fp} (que determina la frecuencia de la señal a ser modulada en la transmisión) y la \gls{fs} (que determina la cantidad de muestras por segundo para la simulación).

Ya que esta simulación trata de la transmisión en \gls{am} a través de un canal telefónico, podemos tomar una \gls{fs} utilizada universalmente en sistemas de audio, y esta equivale a 44 [kHz], y podemos ver que claramente cumple con el Teorema de Nyquist-Shannon ya que es mayor al doble de la señal de mayor frecuencia (1477 [Hz], componente alta de los digitos 3, 6 y 9). Para la \gls{fp} tomamos 15 [kHz] ya que es 10 veces mayor a la señal antes mencionada y es menor a la mitad de \gls{fs} (para poder seguir cumpliendo con el teorema).

A continuación enumeramos el tratamiento de la señal en cada bloque del sistema:

\begin{enumerate}
  \item Codificador DTMF: Se suman las señales sinusoidales correspondientes a las frecuencias que componen cada digito
  \item Modulador AM: Se modula la señal portadora con la señal codificada
  \item Cable Telefónico: La señal modulada pasa por un filtro pasa-banda para simular el canal de voz [300-3300][Hz]
  \item Demodulador AM: Se bate la señal recibida con la misma portadora para obtener la moduladora
  \item Decodificador DTMF: La señal pasa por un banco de filtros pasa-banda para determinar qué señales de la matriz fueron enviadas
\end{enumerate}

\begin{figure}[H]
  \centering
  \includegraphics[width=\linewidth]{images/diagramas-bloques.png}
  \caption{Diagrama de bloques a simular}
  \label{fig:diagrama_bloques_objetivo}
\end{figure}

\section{Decodificación}
Para decodificar la señal del tono se tiene que implementar un banco de filtros en base a la matriz de señales del sistema \gls{dtfm}. Esto es, las filas se corresponden con las frecuencias bajas y las columnas con las frecuencias altas. La sumatoria de las señales se corresponde con la codificación del tono propuesto; esta señal llega a la matriz para devolver el tono correspondiente. El bloque decodificador se compone de filtros pasa-banda para cada frecuencia y el bloque detector como muestra la Figura \ref{fig:diagrama_bloques_decod}. La razón de usar un filtro para cada frecuencia baja y alta, en lugar de un filtro por cada frecuencia resultante, se debe a que de esta forma usamos 7 filtros en lugar de 9 (uno por cada tono); además, estas frecuencias (altas y bajas) están más separadas en el espectro que las frecuencias resultantes, lo cual es provechoso a la hora de diseñar un filtro.

La matriz será un bloque lógico que devolverá el digito correspondiente en base a las señales que hayan logrado activarse luego pasar por el banco de filtros.

\begin{figure}[H]
  \centering
  \includegraphics[width=300pt]{images/diagramas-decodificador.png}
  \caption{Composición del bloque decodificador}
  \label{fig:diagrama_bloques_decod}
\end{figure}

\section{Simulación}
La simulación de este sistema, que es el objetivo de este proyecto, se logrará a través del uso de distintas herramientas dentro del software MATLAB. Entre estas herramientas se encuentra simulink que sirve para realizar el estudio en el tiempo y frecuencia de las señales a través de cada bloque del sistema.

Cada señal de entrada al sistema será simulada a través de generadores de sinusoidales, mientras que los filtros serán bloques que actuen en base a polinomios obtenidos por librerías provistas por MATLAB. El resto de las herramientas serán provistas por simulink (sumadores, multiplicadores, analizadores de espectro, etc.)

La razón de usar una herramienta para el diseño de los filtros es que estos se presumen de alto orden, lo cual es engorroso de calcular de manera analítica. Sin embargo, con el objetivo de comprender cómo es el diseño de filtros digitales, se explicará en el siguiente capítulo cómo es el proceso de diseño y análisis correspondiente.

% Filtros Digitales
\chapter{Filtros Digitales}
\section{Especificaciones}

\section{Diseño}

\section{Conversión de Analógico a Digital}

\section{Análisis}

\section{Conclusiones}

% Desarrollo
\chapter{Desarrollo}
Recordemos de la Figura \ref{fig:diagrama_bloques_objetivo} que nuestro sistema está compuesto por 5 bloques. En este capítulo vamos descomponer y analizar cada bloque para su posterior implementación en la simulación.

\section{Análisis}
\subsection*{Codificador y Modulador}
Estos bloques son en realidad bastante simples. Consta de la sumatoria de dos señales sinusoidales. Para la simulación serían dos generadores establecidos en la combinación de frecuencias que corresponden a un determinado digito. Para crear la codificación, basta con ingresar ambas señales a un sumador. Esto luego debe sumarse a una señal constante de 1, con el fin de poder hacer el producto de la señal resultante con la señal portadora. El resultado de esto es la señal ya modulada, como se muestra en la Figura \ref{fig:bloques_cod_mod}.

\begin{figure}[H]
  \centering
  \includegraphics[width=400pt]{images/diagramas-codificador-modulador.png}
  \caption{Codificador y Modulador}
  \label{fig:bloques_cod_mod}
\end{figure}

\subsection*{Transmisión}
La transmisión será simulada a través de un filtro pasa-banda, estableciendo las frecuencias de corte de tal forma que el ancho de banda sea el espectro audible por el oído humano, como muestra la Figura \ref{fig:bloques_txs}.

\begin{figure}[H]
  \centering
  \includegraphics[width=300pt]{images/diagramas-transmision.png}
  \caption{Transmisión}
  \label{fig:bloques_txs}
\end{figure}

\subsection*{Demodulador}
Para demodular la señal, basta con realizar nuevamente el producto con la señal portadora. Como resultado obtenemos la frecuencia moduladora (que es la señal codificada, suma de las frecuencias que componen a un digito específico) como muestra la Figura \ref{fig:bloques_demod}. Luego debemos aplicar un filtro pasa bajos para limpiar la señal de ruidos que puedan haberse introducido, entre esos, algunos vestigios de la señal portadora.

\begin{figure}[H]
  \centering
  \includegraphics[width=350pt]{images/diagramas-demodulador.png}
  \caption{Transmisión}
  \label{fig:bloques_demod}
\end{figure}

\subsection*{Decodificador}
El bloque decodificador es el más complejo de todos, ya que este tiene la lógica para detectar las señales que componen la señal codificada. Como ya vimos en la Figura \ref{fig:diagrama_bloques_decod}, necesitamos 7 filtros pasa-banda para aislar cada una de las frecuencias de la matriz \gls{dtfm}, luego viene la matriz decodificadora. En la Figura \ref{fig:bloques_decod} vemos una simplificación de cómo estaría compuesta esta lógica de decodificación. Cada salida de control es una compuerta AND que se activara cuando sus dos entradas se encuentren activas (o en "1" lógico). Entonces cada compuerta representa la combinación de tonos para detectar cuál fué el digito enviado.

\begin{figure}[H]
  \centering
  \includegraphics[width=350pt]{images/diagramas-decodificador-matriz.png}
  \caption{Decodificador}
  \label{fig:bloques_decod}
\end{figure}

\section{Diseño}
Como mencionamos en la sección anterior, para poder implementar los filtros digitales del banco de filtros (bloque decodificador) necesitamos librerías de MATLAB para generarlos, ya que estos filtros serán de alto orden lo que es engorroso para el cálculo analítico. Para hacer uso de estas librerías se crearon 2 \textit{scripts} con el objetivo de automatizar la generación de los filtros en base a parámetros de entrada. En el Código \ref{code:banco_decodificador} podemos ver que se encarga de calcular las frecuecias de corte (inferior y superior) para cada frecuencia central provista, bajo una determinada frecuencia de muestreo y orden específico, esto para el banco de filtros del decodificador. Además de eso, también podemos ver que genera las gráficas para analizar la respuesta en frecuencia de cada filtro, como muestra la Figura \ref{fig:banco_filtros_resp_frec}. Luego en el Código \ref{code:algoritmo_principal} tenemos el algorítmo principal, el cual establece las especificaciones generales del sistema, llama a la función para crear el banco de filtros y además crea el resto de los filtros involucrados como el que representa el canal de transmición y el filtro para eleminar la portadora del espectro de trabajo (en la fase de demodulación).

Se puede notar en el Código \ref{code:algoritmo_principal} que establecemos el orden de los filtros en 6 (para pasa-bajos y/o pasa-altos; se interpreta 12 para pasa-banda). Este valor arbitrario, contrario a los cálculos analíticos del capítulo anterior, es empírico; durante varias pruebas de simulación, los filtros pasa-banda de alto orden (20 aproximadamente) demostraban comportamientos inesperados y no concluyentes a la hora de filtrar señales específicas. Por eso, luego de varias pruebas encontramos que el orden 6 era suficiente para realizar la simulación con resultados favorables y realistas.

\begin{figure}[H]
  \lstinputlisting[
    language=Octave,
    caption={Banco Decodificador},
    label={code:banco_decodificador}
  ]{matlab/banco_decodificador.m}
\end{figure}

\begin{figure}[H]
  \centering
  \includegraphics[width=\linewidth]{images/resp_frec_banco_filtros.png}
  \caption{Respuesta en frecuencia del Banco de Filtros}
  \label{fig:banco_filtros_resp_frec}
\end{figure}

\begin{figure}[H]
  \lstinputlisting[
    language=Octave,
    caption={Algorítmo principal},
    label={code:algoritmo_principal}
  ]{matlab/main.m}
\end{figure}

\section{Prototipo}
En base a los resultados obtenidos de los \textit{scripts} diseñamos el sistema completo, en los que cada bloque se alimenta de los datos resultantes. En la Figura \ref{fig:bloques_modem} podemos ver los bloques intervinientes en la primera parte del sistema, esta incluye la selección de los tonos (inferior y superior) para luego sumarlos y lograr la codificación \gls{dtfm} del número 5 en este caso. Luego pasamos a los bloques intervinientes en la modulación AM de las señales, sumando antes una constante 1 para poder realizar el producto con la señal portadora. Una vez obtenida la señal modulada en AM, esta se transmite por el canal de modulación, que según las especificaciones tiene el ancho de banda del espectro audible por el oído humano. Luego llega a la etapa de demodulación, en la que la señal se vuelve a batir (producto) con la portadora, y el resultado es una señal que tiene una componente en la frecuencia de la portadora y debe ser filtrada, por ello utilizamos un filtro pasa-bajos con frecuencia de corte 3[kHz]. Pasado este filtro, la señal obtenida es casi identica a la sumatoria de los dos tonos.

La segunda parte del sistema comprende la decodificación \gls{dtfm} a través de un banco de filtros y una matriz decodificadora, como se muestra en la Figura \ref{fig:bloques_codec}. Primero debemos aislar las señales por tonos diferenciados, esto lo hace el banco de filtros digitales. Cada uno de estos es un filtro pasa-banda con frecuencia central en uno de los 7 tonos, de esta forma logramos aislar cada señal. Dado que estas son sinusoidales, necesitamos calcular el valor efectivo de las mismas para poder operar lógicamente ellas, entonces se coloca un bloque que realiza el cálculo. Luego, el valor obtenido es un número decimal, del cual nos interesa la parte entera, ya que con este dato vamos a validar la amplitud con la que la señal sale del filtro. La razón de hacer este procedimiento es que para poder comparar lógicamente las señales presente para determinar qué tono fue codificado y enviado, lo cual se explica a continuación.

\begin{figure}[H]
  \centering
  \includegraphics[width=\linewidth]{images/modem.png}
  \caption{Codificación, Modulación, Transmisión y Demodulación}
  \label{fig:bloques_modem}
\end{figure}

\begin{figure}[H]
  \centering
  \includegraphics[width=\linewidth]{images/codec.png}
  \caption{Banco de filtros y Decodificador}
  \label{fig:bloques_codec}
\end{figure}

La matriz decodificadora se compone de bloques de operación lógica AND. La salida de este bloque será 1 si y solo si ambas entradas son diferentes a 0\footnote{En el álgebra booleana aplicada en sistemas de control, todo valor igual a 0 se toma como "Falso", mientras que cualquier valor distinto de 0, por más infinitesimal que sea, es "Verdadero"}. Como los filtros anteriores son de orden relativamente bajo, es posible que al tratar las señales dejen pasar vestigios de la otra señal en la codificación pero con mucha menor amplitud. El valor efectivo de este resultado puede ser muy bajo, infinitesimal, del orden de $10^{-2}$, pero no 0 y por consiguiente el operador lógico lo tomará como válido y puede disparar falsos valores. Por esa razón debemos tomar la parte entera del valor eficaz, para garantizar que solo se debe tomar como válida a las señales que realmente son de la misma frecuencia que la frecuencia central de cada filtro.

\section{Simulación}
Para llevar a cabo la simulación vamos a considerar una serie de escenarios posibles, en los que se diferencia mayormente la confiabilidad en el canal de transmisión. Es decir, confiaremos en los bloques de control ya que son implementados a través de sistemas computacionales, pero el medio de transmisión es analógico y su eficacia depende de muchos factores físicos. Este se puede ver alterado de diversas maneras haciendo que parte de la información se pierda o corrompa. Para ello vamos a probar los siguientes escenarios:

\begin{enumerate}
  \item El canal de transmisión es del tipo AWG-24 de menos de 200 [m] de largo, cuyo ancho de banda es de 1 [MHz].
  \item El canal de transimsión es extenso y tiene el ancho de banda del espectro audíble por los humanos.
  \item El mismo canal anterior presenta fallas atenuando en diferentes frecuencias dentro del espectro de tonos \gls{dtfm}
\end{enumerate}

\subsection{Ancho de banda Extendido - 1 [MHz]}
\subsection{Ancho de banda Reducido - 3 [kHz]}
\subsection{Canal con Fallas}

% Conclusiones
% Conclusiones
\chapter{Conclusiones}
\section{Resultados}
Luego de haber completado el desarrollo de la simulación y analizar los resultados obtenidos podemos concluir que es posible implementar este sistema en el mundo real. La decodificación es exitosa para el caso ideal, esto significa que el banco de filtros presenta el comportamiento esperado, dando respuestas correctas y acptables para las diferentes señales emitidas que se encuentren dentro de la banca de paso de cada uno de ellos. Si bien no pudimos utilizar filtros de alto nivel debido a complicaciones con el Simulink, usar filtros de orden 6 fueron suficientes para poder filtrar correctamente las señales de interés.

Así también, probamos que es posible simular un canal ideal, un canal con banda de ancha similar a un cable telefónico y un canal similar pero con fallas, utlizando filtros pasa-banda y rechazabanda. Claramente el canal ideal es lo mejor para la simulación, pero en el mundo real nos tenemos que adaptar e idear soluciones reales, por lo que simular este tipo de caso fue necesario para darnos cuenta que los sistemas de control requieren ajustes para adaptarse a los ambientes reales.

\section{Aplicaciones}
Las posbles aplicaciones para este sistema pueden ser interminables, la premisa es que se requiera un control por teclado númerico y se establezca la comunicación entre un transmisor y receptor que se encuentren relativamente alejados. El medio de transmisión puede variar pero eso se puede ajustar como ya vimos en la simulación. A continuación enumaramos algunos ejemplos.

\subsection*{Sistemas de Alarma y Seguridad}
Se pueden implementar sistemas de alarma que permitan la activación y desactivación remota utilizando tonos DTMF como códigos de acceso. Esos códigos transmitidos por tonos pueden abrir puertas o portones de forma remota, por ejemplo.

\subsection*{Automatización Industrial}
En el ámbito industrial se pueden controlar procesos industriales y maquinaria utilizando señales DTMF para encender, apagar o ajustar parámetros. Por ejemplo, controlar y monitorear vehículos en una flota, como activar alarmas, bloquear o desbloquear puertas, a través de tonos DTMF.

\subsection*{Sistemas de Domótica}
Una aplicación de menor escala puede ser implementar sistemas de automatización en el hogar para controlar luces, termostatos, cortinas, etc., mediante señales DTMF. También se utilizan estas señales para controlar dispositivos como televisores, reproductores de DVD, sistemas de entretenimiento en el hogar, etc.

\section{Problemáticas}
Al implementar cualquier solución a un problema real utilizando este sistema van a surgir complicaciones, y estas están asociadas al ajuste de parámetros o la arquitectura en sí. Es decir, no es lo mismo implementar este sistema para Dómotica que puede tranquilamente mediante la transmisión por aire, que un sistema para automatización industrial que conviene utilizar cables para una transmisión directa. En cada caso, la arquitectura del sistema deberá modificarse. Dependiendo del medio, tal vez sea importante amplificar las señales para compensar la perdida por el medio de transmisión. También hay que considerar que las señales de codificación de mayor frecuencia (tono superior) pueden tener pérdidas en lineas largas de conexión telefónica, por lo que normalmente deberían ser de 1,5\% (2db) mayor con respecto de los tonos bajos.
Otra consideración es que simulink nos limita con respecto al orden de los filtros dentro del banco de filtros. Si esta simulación fuera a través de Matlab únicamente, podríamos implementar filtros de mucho más orden para hacerlos más selectivos, y con ello ajustar mejor los parámetros de redondeo o truncamiento para obtener mejores resultados.

\section{Conceptos}
Además de todo el análisis y desarrollo que llevamos a cabo en este proyecto, debemos destacar otros aspectos del mismo. Hemos desarrollado habilidades que nos permitieron realizar a conciencia este proyecto, tales como el uso de Matlab con sus correspondientes herramientas (Simulink, FDATools, etc.). Estas últimas nos permitieron hacer un diseño y análisis propicio para cada uno de los filtros.

Se pusieron en práctica todos los conocimientos adquiridos durante el cursado de la materia, los cuales nos proporcionaron bases sólidas para que la realización de este proyecto se ejecute de manera eficiente. Esto, además, nos permitió tomar un problema propuesto y llevarlo a la práctica simulada pudiendo obtener los resultados más próximos a los reales. Además estos conocimientos fueron de gran ayuda para resolver los distintos problemas que se nos fueron planteando en el transcurso del desarrollo del banco de filtros digitales.

Este presente proyecto significó un desafío para nosotros, que nos obligó a forjar criterios de ingeniería para desarrollar soluciones a problemas reales.

% Bibliografía
\bibliography{references}

% Glosario: Acrónimos
\printglossaries
\addcontentsline{toc}{chapter}{Siglas}

% Listado de figuras
\listoffigures

% Listado de tablas
\listoftables

% Listado de bloques de código
\lstlistoflistings
\addcontentsline{toc}{chapter}{Índice de bloques de código fuente}

\appendix
% Apendices
\chapter{Cálculos algebráicos}
\section{Diseño de filtro digital H(z)}

\begin{equation}
  \begin{split}
    H(\textrm{z}) & = \frac{BW \left(\frac{2}{T}\frac{(1-\textrm{z}^{-1})}{(1+\textrm{z}^{-1})}\right)}{\left(\frac{2}{T}\frac{(1-\textrm{z}^{-1})}{(1+\textrm{z}^{-1})}\right)^2 + BW \left(\frac{2}{T}\frac{(1-\textrm{z}^{-1})}{(1+\textrm{z}^{-1})}\right) + \Omega_0^2} \\
    H(\textrm{z}) & = \frac{BW \left(\frac{2}{T}\frac{(1-\textrm{z}^{-1})}{(1+\textrm{z}^{-1})}\right)}{\frac{4}{T^2}\frac{(1-\textrm{z}^{-1})^2}{(1+\textrm{z}^{-1})^2} + BW \left(\frac{2}{T}\frac{(1-\textrm{z}^{-1})}{(1+\textrm{z}^{-1})}\right) + \Omega_0^2}\ \frac{(1+\textrm{z}^{-1})^2}{(1+\textrm{z}^{-1})^2} \\
    H(\textrm{z}) & = \frac{BW \frac{2}{T} (1-\textrm{z}^{-1}) (1+\textrm{z}^{-1})}{\frac{4}{T^2} (1-\textrm{z}^{-1})^2+ BW \frac{2}{T} (1-\textrm{z}^{-1}) (1+\textrm{z}^{-1}) + \Omega_0^2 (1+\textrm{z}^{-1})^2} \\
    H(\textrm{z}) & = \frac{2 f_s \ BW  (1-\textrm{z}^{-1}) (1+\textrm{z}^{-1})}{4f_s^2 (1-\textrm{z}^{-1})^2+ 2f_s\ BW (1-\textrm{z}^{-1}) (1+\textrm{z}^{-1}) + \Omega_0^2 (1+\textrm{z}^{-1})^2} \\
    H(\textrm{z}) & = \frac{2 f_s \ BW  (1-\textrm{z}^{-2})}{4f_s^2 (1-2\textrm{z}^{-1}+\textrm{z}^{-2})+ 2f_s\ BW (1-\textrm{z}^{-2}) + \Omega_0^2 (1+2\textrm{z}^{-1}+\textrm{z}^{-2})} \\
    H(\textrm{z}) & = \frac{2 f_s \ BW - 2 f_s \ BW\textrm{z}^{-2}}{\left( 4f_s^2 + 2f_s\ BW + \Omega_0^2 \right) + \left(-8f_s^2 + 2\Omega_0^2\right)\textrm{z}^{-1} + \left( 4f_s^2 - 2f_s\ BW + \Omega_0^2 \right)\textrm{z}^{-2}} \\
    \text{Donde} & \\
    \alpha & = 4f_s^2 + 2f_s\ BW + \Omega_0^2 \\
    \beta & = 2\Omega_0^2 - 8f_s^2 \\
    \gamma & = 4f_s^2 - 2f_s\ BW + \Omega_0^2 \\
    \delta & = 2 f_s \ BW \\
    \text{Luego} & \\
    H(\textrm{z}) & = \frac{\delta - \delta \textrm{z}^{-2}}{\alpha + \beta\textrm{z}^{-1} + \gamma\textrm{z}^{-2}} \\
    H(\textrm{z}) & = \frac{\frac{\delta}{\alpha} - \frac{\delta}{\alpha} \textrm{z}^{-2}}{1 + \frac{\beta}{\alpha} \textrm{z}^{-1} + \frac{\gamma}{\alpha} \textrm{z}^{-2}} \\
    H(\textrm{z}) & = \frac{8,4953\times 10^{-3} - 8,4953\times 10^{-3} \textrm{z}^{-2}}{1 - 1,9732 \textrm{z}^{-1} + 0,9830 \textrm{z}^{-2}} \\
  \end{split}
  \label{eq:apendix_filtro_digital}
\end{equation}

\chapter{Scripts de Matlab}
\section{Polos y Ceros}
\begin{figure}[H]
  \lstinputlisting[
    language=Octave,
    caption={Graficar Polos y Ceros para Plano de Laplace y Z},
    label={code:script_pz}
  ]{matlab/filtros-digitales/plot_pz.m}
\end{figure}

\section{Respuesta al Impulso y al Escalón}
\begin{figure}[H]
  \lstinputlisting[
    language=Octave,
    caption={Graficar respuesta al Impulso y al Escalón del filtro Digital},
    label={code:script_impulse_step}
  ]{matlab/filtros-digitales/plot_impulse_step.m}
\end{figure}

\section{Estabilidad de Filtro Pasa-Banda}
\begin{figure}[H]
  \lstinputlisting[
    language=Octave,
    caption={Análisis de Estabilidad por Filtro Pasa-Banda},
    label={code:script_analisis}
  ]{matlab/desarrollo/analisis_filtro.m}
\end{figure}
\chapter{Análisis de Estabilidad}
\pagebreak

\section{Filtro pasa-banda centrado en 697 [Hz]}
\begin{figure}[H]
  \centering
  \includegraphics[width=\linewidth]{images/simulacion/697.png}
  \caption{Análisis de estabilidad para el filtro centrado en 697 [Hz]}
  \label{fig:analisis_697}
\end{figure}

\section{Filtro pasa-banda centrado en 770 [Hz]}
\begin{figure}[H]
  \centering
  \includegraphics[width=\linewidth]{images/simulacion/770.png}
  \caption{Análisis de estabilidad para el filtro centrado en 770 [Hz]}
  \label{fig:analisis_770}
\end{figure}

\section{Filtro pasa-banda centrado en 852 [Hz]}
\begin{figure}[H]
  \centering
  \includegraphics[width=\linewidth]{images/simulacion/852.png}
  \caption{Análisis de estabilidad para el filtro centrado en 852 [Hz]}
  \label{fig:analisis_852}
\end{figure}

\section{Filtro pasa-banda centrado en 941 [Hz]}
\begin{figure}[H]
  \centering
  \includegraphics[width=\linewidth]{images/simulacion/941.png}
  \caption{Análisis de estabilidad para el filtro centrado en 941 [Hz]}
  \label{fig:analisis_941}
\end{figure}

\section{Filtro pasa-banda centrado en 1209 [Hz]}
\begin{figure}[H]
  \centering
  \includegraphics[width=\linewidth]{images/simulacion/1209.png}
  \caption{Análisis de estabilidad para el filtro centrado en 1209 [Hz]}
  \label{fig:analisis_1209}
\end{figure}

\section{Filtro pasa-banda centrado en 1336 [Hz]}
\begin{figure}[H]
  \centering
  \includegraphics[width=\linewidth]{images/simulacion/1336.png}
  \caption{Análisis de estabilidad para el filtro centrado en 1336 [Hz]}
  \label{fig:analisis_1336}
\end{figure}

\section{Filtro pasa-banda centrado en 1477 [Hz]}
\begin{figure}[H]
  \centering
  \includegraphics[width=\linewidth]{images/simulacion/1477.png}
  \caption{Análisis de estabilidad para el filtro centrado en 1477 [Hz]}
  \label{fig:analisis_1477}
\end{figure}

\pagebreak

\section{Polos y Ceros}
\label{sec:pz}
\subsection*{Filtro centrado en 697 [Hz]}
\begin{multicols}{2}
  \subsubsection*{Polos}
  $0.9916 + 0.1039i$\\
  $0.9916 - 0.1039i$\\
  $0.9928 + 0.0941i$\\
  $0.9928 - 0.0941i$\\
  $0.9894 + 0.0985i$\\
  $0.9894 - 0.0985i$\\

  \columnbreak
  \subsubsection*{Ceros}
  $-1.0000 + 0.0000i$\\
  $-1.0000 - 0.0000i$\\
  $-1.0000 + 0.0000i$\\
  $1.0000 + 0.0000i$\\
  $1.0000 + 0.0000i$\\
  $1.0000 - 0.0000i$\\
\end{multicols}

\subsection*{Filtro centrado en 770 [Hz]}
\begin{multicols}{2}
  \subsubsection*{Polos}
  $0.9905 + 0.1142i$\\
  $0.9905 - 0.1142i$\\
  $0.9918 + 0.1045i$\\
  $0.9918 - 0.1045i$\\
  $0.9883 + 0.1088i$\\
  $0.9883 - 0.1088i$\\

  \columnbreak
  \subsubsection*{Ceros}
  $-1.0000 + 0.0000i$\\
  $-1.0000 - 0.0000i$\\
  $-1.0000 + 0.0000i$\\
  $1.0000 + 0.0000i$\\
  $1.0000 + 0.0000i$\\
  $1.0000 - 0.0000i$\\
\end{multicols}

\subsection*{Filtro centrado en 852 [Hz]}
\begin{multicols}{2}
  \subsubsection*{Polos}
  $0.9891 + 0.1258i$\\
  $0.9891 - 0.1258i$\\
  $0.9905 + 0.1161i$\\
  $0.9905 - 0.1161i$\\
  $0.9870 + 0.1204i$\\
  $0.9870 - 0.1204i$\\

  \columnbreak
  \subsubsection*{Ceros}
  $-1.0000 + 0.0000i$\\
  $-1.0000 - 0.0000i$\\
  $-1.0000 + 0.0000i$\\
  $1.0000 + 0.0000i$\\
  $1.0000 + 0.0000i$\\
  $1.0000 - 0.0000i$\\
\end{multicols}

\pagebreak

\subsection*{Filtro centrado en 941 [Hz]}
\begin{multicols}{2}
  \subsubsection*{Polos}
  $0.9874 + 0.1384i$\\
  $0.9874 - 0.1384i$\\
  $0.9889 + 0.1287i$\\
  $0.9889 - 0.1287i$\\
  $0.9854 + 0.1330i$\\
  $0.9854 - 0.1330i$\\

  \columnbreak
  \subsubsection*{Ceros}
  $-1.0000 + 0.0000i$\\
  $-1.0000 - 0.0000i$\\
  $-1.0000 + 0.0000i$\\
  $1.0000 + 0.0000i$\\
  $1.0000 + 0.0000i$\\
  $1.0000 - 0.0000i$\\
\end{multicols}

\subsection*{Filtro centrado en 1209 [Hz]}
\begin{multicols}{2}
  \subsubsection*{Polos}
  $0.9814 + 0.1761i$\\
  $0.9814 - 0.1761i$\\
  $0.9832 + 0.1664i$\\
  $0.9832 - 0.1664i$\\
  $0.9796 + 0.1706i$\\
  $0.9796 - 0.1706i$\\

  \columnbreak
  \subsubsection*{Ceros}
  $-1.0000 + 0.0000i$\\
  $-1.0000 + 0.0000i$\\
  $-1.0000 - 0.0000i$\\
  $1.0000 + 0.0000i$\\
  $1.0000 - 0.0000i$\\
  $1.0000 + 0.0000i$\\
\end{multicols}

\subsection*{Filtro centrado en 1336 [Hz]}
\begin{multicols}{2}
  \subsubsection*{Polos}
  $0.9780 + 0.1939i$\\
  $0.9780 - 0.1939i$\\
  $0.9801 + 0.1842i$\\
  $0.9801 - 0.1842i$\\
  $0.9763 + 0.1884i$\\
  $0.9763 - 0.1884i$\\

  \columnbreak
  \subsubsection*{Ceros}
  $-1.0000 + 0.0000i$\\
  $-1.0000 - 0.0000i$\\
  $-1.0000 + 0.0000i$\\
  $1.0000 + 0.0000i$\\
  $1.0000 + 0.0000i$\\
  $1.0000 - 0.0000i$\\
\end{multicols}

\subsection*{Filtro centrado en 1477 [Hz]}
\begin{multicols}{2}
  \subsubsection*{Polos}
  $0.9740 + 0.2135i$\\
  $0.9740 - 0.2135i$\\
  $0.9761 + 0.2039i$\\
  $0.9761 - 0.2039i$\\
  $0.9723 + 0.2080i$\\
  $0.9723 - 0.2080i$\\

  \columnbreak
  \subsubsection*{Ceros}
  $-1.0000 + 0.0000i$\\
  $-1.0000 - 0.0000i$\\
  $-1.0000 + 0.0000i$\\
  $1.0000 + 0.0000i$\\
  $1.0000 + 0.0000i$\\
  $1.0000 - 0.0000i$\\
\end{multicols}


\end{document}
