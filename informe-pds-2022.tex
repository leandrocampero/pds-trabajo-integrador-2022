\documentclass[a4paper]{report}

% Preamble
\usepackage[spanish]{babel} % Paquete para utilizar el idioma español
\usepackage[utf8]{inputenc} % Codificación de caracteres UTF-8
\usepackage{setspace} % Para ajustar el espaciado entre líneas
\usepackage{titlesec} % Para personalizar los encabezados de capítulos y secciones
\usepackage{lipsum} % Para generar texto ficticio, puedes eliminar esta línea
\usepackage{enumitem} % Para enumerar items
\usepackage[hidelinks]{hyperref} % Para generar los enlaces a las secciones correspondientes
\usepackage{geometry}
\usepackage{graphicx}
\usepackage{fancyhdr}
\usepackage{microtype}
\usepackage{hyperref}
\usepackage{array}
\usepackage{chngcntr}
\usepackage{tocbibind}

% Otras configuraciones del preámbulo

% Header y footer
\pagestyle{fancy}
\fancyhf{}
\lhead{Procesamiento Digital de Señales}
\rhead{Trabajo Integrador}
\lfoot{Boeri - Campero - Villafañe | Ingeniería en Computación}
\rfoot{\thepage}

\fancypagestyle{plain}{
  \fancyhf{}
  \lhead{Procesamiento Digital de Señales}
  \rhead{Trabajo Integrador}
  \lfoot{Boeri - Campero - Villafañe | Ingeniería en Computación}
  \rfoot{\thepage}
}

\fancyhfoffset[R]{0pt} % Ajusta el ancho del encabezado
\fancyhfoffset[L]{0pt} % Ajusta el ancho del pie de página

\newgeometry{
  top=1.25in,
  bottom=1in,
  outer=0.75in,
  inner=0.75in,
}


% Configuración de página
\setlength{\parindent}{0pt} % Deshabilitar sangría al inicio de párrafos
\setlength{\parskip}{1em} % Establecer espaciado entre párrafos

% Personalizaciones
\titleformat{\chapter}[display]
{\normalfont\huge\bfseries}{\chaptertitlename\ \thechapter}{20pt}{\Huge}
\titlespacing*{\chapter}{0pt}{0pt}{40pt} % Ajustar espaciado antes y después de los encabezados de capítulo

\renewcommand{\normalsize}{\fontsize{14}{18}\selectfont}

\newcommand{\titulo}{Sistema de Transmisión AM con Codificación DTMF}
\newcommand{\subtitulo}{Informe de Trabajo Integrador}
\newcommand{\materia}{Procesamiento Digital de Señales}
\newcommand{\departamento}{Cátedra de Procesamiento de Señales}
\newcommand{\unidadacademica}{Facultad de Ciencias Exactas y Tecnología}
\newcommand{\universidad}{Universidad Nacional de Tucumán}
\newcommand{\autores}{
  Boeri, Benjamin \\
  Campero, Leandro \\
  Villafañe, Cristian
}
\newcommand{\fecha}{\today}

% Comienza el documento
\begin{document}

% Página de título
\begin{titlepage}
  \centering
  \vspace*{1cm}

  \begin{spacing}{2}
    \textbf{\Huge \titulo}
  \end{spacing}

  \Large \subtitulo

  \vspace{1cm}

  \textbf{\Large \materia}

  \vspace{0.5cm}

  \textbf{\Large \departamento}

  \vspace{0.5cm}

  \textbf{\Large \unidadacademica}

  \vspace{0.5cm}

  \textbf{\Large \universidad}

  \vspace{2cm}

  \textbf{\Large Autores:}

  \autores

  \vfill

  \Large \fecha

\end{titlepage}

% Resumen
\begin{abstract}
  Aquí va el resumen.
\end{abstract}

% Índice
\tableofcontents

% Introduccion
% Capítulo 1
\chapter{Introducción}
\section{Problema propuesto}
La modulación en amplitud o \gls{am}, permite la transmisión de una señal de baja frecuencia superpuesta a una onda de alta frecuencia. Este sistema de modulación permite enviar mensajes en la forma de envolventes de la onda portadora, ya sea por un canal de aire o físico utilizando un enlace cableado.\\
El sistema de codificación \gls{dtfm}, utiliza una combinación de tonos de frecuencia audibles pera representar el conjunto de números del 0 al 9 disponible en el teclado telefónico, con lo cual es posible enviar una codificación numérica por la línea telefónica.\\
El modelo de trabajo está representado en la Figura \ref{fig:intro_diagrama_bloques}, correspondientes al Modulador y Demodulador \gls{am}, el canal de cable telefónico, y las etapas de codificación y decodificación \gls{dtfm}.

\begin{figure}[ht]
  \centering
  \includegraphics[width=\linewidth]{images/planteamiento/bloques.png}
  \caption{Diagrama de bloques general}
  \label{fig:intro_diagrama_bloques}
\end{figure}

\section{Objetivo}
El objetivo principal de este proyecto integrador es la implementación del sistema mostrado en la Figura \ref{fig:intro_diagrama_bloques}, utilizando MATLAB, SIMULINK, o la combinación de ambos recursos de modelado computacional, para el envío de números (0-9) codificados en \gls{dtfm} bajo modulación \gls{am}, y la detección del número enviado a la salida (uno número cada por vez).\\
A modo de referencia, el Cuadro \ref{tab:combinacion_tonos}, muestra la combinación de tonos audibles asociados al conjunto numérico, y en el enlace indicado se encuentra la información ampliada sobre la codificación \gls{dtfm}.

\begin{table}[H]
  \centering
  \begin{tabular}{|Sc|Sc|Sc|Sc|}
    \hline
    \textbf{Frecuencia Baja} & \textbf{Frecuencia Alta} & \textbf{Digito} & \textbf{Frecuencia Final} \\ \hline
    697                      & 1209                     & 1               & 1906                      \\ \hline
    697                      & 1336                     & 2               & 2033                      \\ \hline
    697                      & 1477                     & 3               & 2174                      \\ \hline
    770                      & 1209                     & 4               & 1979                      \\ \hline
    770                      & 1336                     & 5               & 2106                      \\ \hline
    770                      & 1477                     & 6               & 2247                      \\ \hline
    852                      & 1209                     & 7               & 2061                      \\ \hline
    852                      & 1336                     & 8               & 2188                      \\ \hline
    852                      & 1477                     & 9               & 2329                      \\ \hline
    941                      & 1336                     & 0               & 2277                      \\ \hline
  \end{tabular}
  \caption{Combinación de tonos audibles (medido en [Hz])}
  \label{tab:combinacion_tonos}
\end{table}

\section{Enunciado}
\begin{enumerate}[label=\alph*)]
  \item A nivel simulación se deberán sintetizar los tonos asociados a cada digito numérico seleccionando una \gls{fs} apropiada (Teorema de Nyquist-Shannon).
  \item El demodulador \gls{dtfm} deberá ser implementado mediante filtros digitales pasa bandas, con un orden y respuestas apropiadas. El modo de indicar cuál fue el digito enviado queda a criterio del grupo de trabajo.
  \item Para el modelo de trasmisión AM (enlace cableado) se deberán establecer y sintetizar la frecuencia de portadora $RF$ el índice de modulación apropiados (recordando que \gls{fs} es única en todo el sistema).
  \item El canal de transmisión se corresponde al de un filtro analógico (transformado a digital) pasa banda con un rango de 300 Hz a 3400 Hz, respuesta plana y orden apropiado. Se considera el rango útil asignado a la frecuencia telefónica, aunque el cable         telefónico de cobre tipo AWG-24, por ejemplo, supera este ancho de banda a 1Mz en distancias inferiores a 200 Mts.
\end{enumerate}

\section{Lineamientos Generales}
\begin{enumerate}[label=\alph*)]
  \item El grupo de trabajo deberá cumplir con las especificaciones del proyecto, utilizando criterios de diseños justificados para cada bloque del sistema.
  \item Se deberán indicar el paso a paso para el diseño de los filtros digitales utilizados en las diferentes etapas.
  \item El criterio de selección para el filtro analógico representativo del canal (Bessel, Butterworth, etc.), y el método de transformación analógico a discreto escogido, brindando una gráfica comparativa de la respuesta en frecuencia resultantes en ambos planos (Laplace y Z).
  \item Se pide 3 aplicaciones posibles del sistema desarrollado en aplicaciones de tele comando (por ejemplo, aplicación de sistema de riego por comando telefónico de 3 zonas), y como se imprentaría en la práctica (no el desarrollo, solo la propuesta).
  \item Problema de análisis: para el caso de que ocurran fallos en el canal de comunicación (por ejemplo, una atenuación en determinadas frecuencias), analizar la robustez del código detector para al menos 3 zonas atenuadas de frecuencias diferentes. Utilizar el código adjunto en Matlab para el diseño del canal con fallas. Justificar los resultados.
  \item Escribir el informe, Incluir conclusiones, observaciones y sugerencias sobre los resultados obtenidos
\end{enumerate}

\section{Instrucciones de simulación}
Para poder reproducir las simulaciones expuestas en este informe hay que tener en cuenta los siguientes archivos.

\subsection*{Scripts}
Dentro del directorio matlab se encuentran los scripts de las simulaciones. Existen subdirectorios que contienen algunas de las librerías propias utilizadas. A continuación se describen los archivos dentro del directorio:
\begin{enumerate}
  \item \textbf{\lstinline{sim_ideal.m}}: ejecutar este script para generar el banco de filros más el canal ideal (mayor ancho de banda).
  \item \textbf{\lstinline{sim_real.m}}: ejecutar este script para generar el banco de filros más el canal real (ancho de banda reducido, como de un canal telefónico).
  \item \textbf{\lstinline{sim_falla.m}}: ejecutar este script para generar el banco de filros más el canal con fallas.
  \item \textbf{\lstinline{analisis.m}}: script utilizado para analizar uno por uno los prototipos digitales del banco de filtros.
  \item \textbf{\lstinline{canal.m}}: script utilizado para analizar el filtro representativo del canal de transmisión. Por defecto analiza el canal con fallas pero se puede modificar.
\end{enumerate}

\subsection*{Simulink}
Dentro del directorio matlab se encuentran los archivos del Simulink, en los que se desarrolla el sistema. Existen dos archivos:
\begin{enumerate}
  \item \textbf{\lstinline{simulacion_completa.slx}}: sistema para simular toda la arquitectura completa. Contiene los selectores de tonos más un selector de canal utilizado en la simulación de canal real (telefónico).
  \item \textbf{\lstinline{simulacion_sin_am.slx}}: sistema para simular la arquitectura sin la modulación y demodulación en amplitud (más adelante se detalla la razón de esta modificación, en el Capítulo \ref{cap:desarrollo}). Contiene los mismos selectores, solo que el del canal tiene para alternar entre canal amplificado y no amplificado.
\end{enumerate}

\chapter{Revisión de literatura}
\section{Estudios anteriores}
Este capítulo contiene tu revisión de literatura.

\chapter{Metodología}
\section{Diseño de investigación}
Aquí explicas tu metodología de investigación.

% Añade más capítulos y secciones según sea necesario

% Conclusiones
\chapter{Conclusiones}
\section{Resumen de resultados}
Tu capítulo de conclusiones comienza aquí.

% Bibliografía
\begin{thebibliography}{}
  % Incluye tus referencias aquí
\end{thebibliography}

% Listado de figuras
\listoffigures

% Listado de tablas
\listoftables

% Apendices
\include{chapters/apendices}

\end{document}
