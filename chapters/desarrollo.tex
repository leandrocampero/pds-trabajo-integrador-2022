\chapter{Desarrollo}
Recordemos de la Figura \ref{fig:diagrama_bloques_objetivo} que nuestro sistema está compuesto por 5 bloques. En este capítulo vamos descomponer y analizar cada bloque para su posterior implementación en la simulación.

\section{Análisis}
\subsection*{Codificador y Modulador}
Estos bloques son en realidad bastante simples. Consta de la sumatoria de dos señales sinusoidales. Para la simulación serían dos generadores establecidos en la combinación de frecuencias que corresponden a un determinado digito. Para crear la codificación, basta con ingresar ambas señales a un sumador. Esto luego debe sumarse a una señal constante de 1, con el fin de poder hacer el producto de la señal resultante con la señal portadora. El resultado de esto es la señal ya modulada, como se muestra en la Figura \ref{fig:bloques_cod_mod}.

\begin{figure}[H]
  \centering
  \includegraphics[width=400pt]{images/diagramas-codificador-modulador.png}
  \caption{Codificador y Modulador}
  \label{fig:bloques_cod_mod}
\end{figure}

\subsection*{Transmisión}
La transmisión será simulada a través de un filtro pasa-banda, estableciendo las frecuencias de corte de tal forma que el ancho de banda sea el espectro audible por el oído humano, como muestra la Figura \ref{fig:bloques_txs}.

\begin{figure}[H]
  \centering
  \includegraphics[width=300pt]{images/diagramas-transmision.png}
  \caption{Transmisión}
  \label{fig:bloques_txs}
\end{figure}

\subsection*{Demodulador}
Para demodular la señal, basta con realizar nuevamente el producto con la señal portadora. Como resultado obtenemos la frecuencia moduladora (que es la señal codificada, suma de las frecuencias que componen a un digito específico) como muestra la Figura \ref{fig:bloques_demod}. Luego debemos aplicar un filtro pasa bajos para limpiar la señal de ruidos que puedan haberse introducido, entre esos, algunos vestigios de la señal portadora.

\begin{figure}[H]
  \centering
  \includegraphics[width=350pt]{images/diagramas-demodulador.png}
  \caption{Transmisión}
  \label{fig:bloques_demod}
\end{figure}

\subsection*{Decodificador}
El bloque decodificador es el más complejo de todos, ya que este tiene la lógica para detectar las señales que componen la señal codificada. Como ya vimos en la Figura \ref{fig:diagrama_bloques_decod}, necesitamos 7 filtros pasa-banda para aislar cada una de las frecuencias de la matriz \gls{dtfm}, luego viene la matriz decodificadora. En la Figura \ref{fig:bloques_decod} vemos una simplificación de cómo estaría compuesta esta lógica de decodificación. Cada salida de control es una compuerta AND que se activara cuando sus dos entradas se encuentren activas (o en "1" lógico). Entonces cada compuerta representa la combinación de tonos para detectar cuál fué el digito enviado.

\begin{figure}[H]
  \centering
  \includegraphics[width=350pt]{images/diagramas-decodificador-matriz.png}
  \caption{Decodificador}
  \label{fig:bloques_decod}
\end{figure}


\section{Diseño}
Como mencionamos en la sección anterior, para poder implementar los filtros digitales del banco de filtros (bloque decodificador) necesitamos librerías de MATLAB para generarlos, ya que estos filtros serán de alto orden lo que es engorroso para el cálculo analítico. Para hacer uso de estas librerías se crearon 2 \textit{scripts} con el objetivo de automatizar la generación de los filtros en base a parámetros de entrada. En el Código \ref{code:banco_decodificador} podemos ver que se encarga de calcular las frecuecias de corte (inferior y superior) para cada frecuencia central provista, bajo una determinada frecuencia de muestreo y orden específico, esto para el banco de filtros del decodificador. Luego en el Código \ref{code:algoritmo_principal} tenemos el algorítmo principal, el cual establece las especificaciones generales del sistema, llama a la función para crear el banco de filtros y además crea el resto de los filtros involucrados como el que representa el canal de transmición y el filtro para eleminar la portadora del espectro de trabajo (en la fase de demodulación).

Se puede notar en el Código \ref{code:algoritmo_principal} que establecemos el orden de los filtros en 6 (para pasa-bajos y/o pasa-altos; se interpreta 12 para pasa-banda). Este valor arbitrario, contrario a los cálculos analíticos del capítulo anterior, es empírico; durante varias pruebas de simulación, los filtros pasa-banda de alto orden (20 aproximadamente) demostraban comportamientos inesperados y no concluyentes a la hora de filtrar señales específicas. Por eso, luego de varias pruebas encontramos que el orden 6 era suficiente para realizar la simulación con resultados favorables y realistas.



\begin{figure}[H]
  \lstinputlisting[
    language=Octave,
    caption={Banco Decodificador},
    label={code:banco_decodificador}
  ]{matlab/banco_decodificador.m}
\end{figure}

\begin{figure}[H]
  \lstinputlisting[
    language=Octave,
    caption={Algorítmo principal},
    label={code:algoritmo_principal}
  ]{matlab/main.m}
\end{figure}

En base a los resultados obtenidos de los \textit{scripts} diseñamos el sistema completo, en los que cada bloque se alimenta de los datos resultantes. (Continuar que no queda nada)


\begin{figure}[H]
  \centering
  \includegraphics[width=\linewidth]{images/modem.png}
  \caption{Codificación, Modulación, Transmisión y Democulación}
  \label{fig:bloques_modem}
\end{figure}


\begin{figure}[H]
  \centering
  \includegraphics[width=\linewidth]{images/codec.png}
  \caption{Banco de filtros y Decodificador}
  \label{fig:bloques_codec}
\end{figure}


\section{Especificaciones}

\section{Prototipo}