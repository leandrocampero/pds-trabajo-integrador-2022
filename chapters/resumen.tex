\begin{abstract}
  El procesamiento digital de señales se refiere a la manipulación, análisis y transformación de señales utilizando algoritmos y técnicas computacionales. En este proyecto, el procesamiento digital de señales se aplica para generar y decodificar los tonos DTMF, así como para simular la transmisión y detección de los dígitos enviados.

  Este proyecto presenta la implementación de un sistema de Modulación en Amplitud (AM) y Codificación DTMF utilizando MATLAB/SIMULINK. Se busca transmitir dígitos numéricos codificados en DTMF a través de un enlace cableado simulado. El sistema involucra la generación de tonos DTMF, el diseño de filtros digitales pasa bandas para el decodificador, la configuración de la frecuencia de portadora RF y la modelización del canal de transmisión como un filtro analógico pasa banda.

  Un filtro digital es un componente esencial en el procesamiento digital de señales que permite modificar las características de una señal. En este proyecto, se utilizan filtros digitales pasa bandas implementados mediante la técnica del filtro Butterworth. Estos filtros permiten seleccionar y aislar las frecuencias específicas asociadas a los tonos DTMF.

  La implementación de los filtros digitales se logra mediante la transformación de los coeficientes del filtro en una representación numérica que se aplica a la señal de entrada. Esto puede lograrse utilizando algoritmos y técnicas de programación, así como también herramientas como MATLAB y SIMULINK.

  El resultado de este proyecto demuestra la viabilidad y efectividad de la implementación del sistema propuesto. El procesamiento digital de señales y el uso de filtros digitales son fundamentales en diversas aplicaciones, incluyendo las comunicaciones y el procesamiento de señales de audio.

  En resumen, este proyecto combina el procesamiento digital de señales, la modulación AM, el diseño de filtros digitales y la codificación DTMF para lograr la transmisión y detección de dígitos numéricos. La implementación exitosa de este sistema contribuye al avance y comprensión de las técnicas de procesamiento y transmisión de señales en el ámbito de las comunicaciones.
\end{abstract}
