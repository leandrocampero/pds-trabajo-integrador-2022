%================================================%
%                                                %
%              Portada y meta datos              %
%                                                %
%================================================%

\hypersetup{
  pdftitle={Sistema de Transmisión AM con Codificación DTMF},
  pdfauthor={Boeri, Benjamin; Campero, Leandro; Villafañe, Cristian},
  pdfsubject={Informe de Trabajo Integrador - Procesamiento Digital de Señales 2022},
  pdfkeywords={PDS, Informe, Boeri, Campero, Villafañe, 2022},
  pdfproducer={LaTeX},
  pdfcreator={pdfLaTeX}
}

\newcommand{\titulo}{Sistema de Transmisión AM con Codificación DTMF}
\newcommand{\subtitulo}{Informe de Trabajo Integrador}
\newcommand{\materia}{Procesamiento Digital de Señales}
\newcommand{\carrera}{Ingeniería en Computación}
\newcommand{\unidadacademica}{Facultad de Ciencias Exactas y Tecnología}
\newcommand{\universidad}{Universidad Nacional de Tucumán}
\newcommand{\autores}{
  Boeri, Benjamin \\
  Campero, Leandro \\
  Villafañe, Cristian
}
\newcommand{\fecha}{\today}

%================================================%
%                                                %
%           Encabezado y Pié de página           %
%                                                %
%================================================%

\pagestyle{fancy}
\fancyhf{}
\lhead{Procesamiento Digital de Señales}
\rhead{Boeri - Campero - Villafañe}
\lfoot{\nouppercase{\leftmark}}
\rfoot{\thepage}

\fancypagestyle{plain}{
  \fancyhf{}
  \lhead{Procesamiento Digital de Señales}
  \rhead{Boeri - Campero - Villafañe}
  \lfoot{\nouppercase{\leftmark}}
  \rfoot{\thepage}
}

%================================================%
%                                                %
%            Configuración de página             %
%                                                %
%================================================%

\newgeometry{
  top=1.25in,
  bottom=1in,
  outer=0.75in,
  inner=0.75in,
}

\setmainfont{Times New Roman} % Fuente del documento
\renewcommand{\normalsize}{\fontsize{12}{14}\selectfont} % Tamaño de fuente e interlineado
\setlength{\parindent}{0pt} % Deshabilitar sangría al inicio de párrafos
\setlength{\parskip}{5pt} % Establecer espaciado entre párrafos

\fancyhfoffset[R]{0pt} % Ajusta el ancho del encabezado
\fancyhfoffset[L]{0pt} % Ajusta el ancho del pie de página

% Personalizar titulos
\titleformat{\chapter}[display]
{\normalfont\huge\bfseries}
{\chaptertitlename\ \thechapter}{20pt}{\Huge}
\titlespacing*{\chapter}{0pt}{0pt}{40pt} % Ajustar espaciado antes y después de los encabezados de capítulo

% Quitar quiebre de palabras
\tolerance=1
\emergencystretch=\maxdimen
\hyphenpenalty=1000
\hbadness=1000

\counterwithout{table}{section} % Desvincular el contador de las tablas de las secciones
\counterwithout{figure}{section} % Desvincular el contador de las figuras de las secciones
\counterwithin{table}{chapter} % Vincular el contador de las tablas al contador de los capítulos
\counterwithin{figure}{chapter} % Vincular el contador de las figuras al contador de los capítulos

\bibliographystyle{alpha}

%================================================%
%                                                %
%                     Tablas                     %
%                                                %
%================================================%

\setlength{\tabcolsep}{10pt} % Espaciado horizontal
\setlength\cellspacetoplimit{7pt} % Espaciado Superior
\setlength\cellspacebottomlimit{7pt} % Espaciado Inferior

%================================================%
%                                                %
%            Bloques de código fuente            %
%                                                %
%================================================%

\definecolor{lightbluegray}{HTML}{eeeeee} % Define el color de fondo
\lstset{
  backgroundcolor=\color{lightbluegray}, % Establece el color de fondo
  basicstyle=\fontsize{10}{12}\ttfamily,
  keywordstyle=\color{blue}\bfseries,
  commentstyle=\color{gray}\itshape,
  numberstyle=\color{gray},
  stringstyle=\color{orange},
  captionpos=b,
  numbers=left,
  breaklines=true,
  frame=single,
  keepspaces=true,
  showspaces=false,
  showstringspaces=false,
  numbersep=8pt,
  xleftmargin=16pt,
  aboveskip=24pt
}
\renewcommand\lstlistingname{Código fuente}
\renewcommand\lstlistlistingname{Índice de bloques de código fuente}

%================================================%
%                                                %
%                   Acrónimos                    %
%                                                %
%================================================%

\newacronym{dtfm}{DTFM}{Doble Tonos Múltiples Frecuencias}
\newacronym{am}{AM}{Amplitud Modulada}
\newacronym{ab}{AB}{Ancho de Banda}
\newacronym{fs}{$f_s$}{Frecuencia de Muestreo}
\newacronym{fc}{$f_c$}{Frecuencia de Corte}
\newacronym{fa}{$f_a$}{Frecuencia de Atenuación o Paso}
\newacronym{fp}{$f_p$}{Frecuencia de Portadora}
\makeglossaries