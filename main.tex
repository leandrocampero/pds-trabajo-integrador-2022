\documentclass{report}

% Preamble
\usepackage[spanish]{babel} % Paquete para utilizar el idioma español
\usepackage[utf8]{inputenc} % Codificación de caracteres UTF-8
\usepackage{setspace} % Para ajustar el espaciado entre líneas
\usepackage{titlesec} % Para personalizar los encabezados de capítulos y secciones
\usepackage{lipsum} % Para generar texto ficticio, puedes eliminar esta línea

% Configuración de página
\setlength{\parindent}{0pt} % Deshabilitar sangría al inicio de párrafos
\setlength{\parskip}{1em} % Establecer espaciado entre párrafos

% Personalizaciones
\titleformat{\chapter}[display]
{\normalfont\huge\bfseries}{\chaptertitlename\ \thechapter}{20pt}{\Huge}
\titlespacing*{\chapter}{0pt}{-50pt}{40pt} % Ajustar espaciado antes y después de los encabezados de capítulo

% Comienza el documento
\begin{document}

% Página de título
\title{Título de tu tesis}
\author{Tú Nombre}
\date{\today}
\maketitle

% Resumen
\begin{abstract}
  Aquí va el resumen.
\end{abstract}

% Índice
\tableofcontents

% Capítulos
\chapter{Introducción}
\section{Antecedentes}
Aquí comienza tu capítulo de introducción.

\chapter{Revisión de literatura}
\section{Estudios anteriores}
Este capítulo contiene tu revisión de literatura.

\chapter{Metodología}
\section{Diseño de investigación}
Aquí explicas tu metodología de investigación.

% Añade más capítulos y secciones según sea necesario

% Conclusiones
\chapter{Conclusiones}
\section{Resumen de resultados}
Tu capítulo de conclusiones comienza aquí.

% Bibliografía
\begin{thebibliography}{}
  % Incluye tus referencias aquí
\end{thebibliography}

\end{document}
